\chapter{Cos'è una Smart City}
L'enciclopedia Treccani definisce una Smart City una \textit{"Città caratterizzata dall’integrazione tra saperi, strutture e mezzi tecnologicamente avanzati, propri della società della comunicazione e dell’informazione, finalizzati a una crescita sostenibile e al miglioramento della qualità della vita."}\cite{definizione_smart_city}
Molte società, non solo multinazionali, stanno investendo in questo settore. Secondo uno studio del Center for Technology in Government dell'University at Albany, The State University of New York, le più grandi banche di investimento mondiali prevedono di investire oltre \$ 175 miliardi di dollari in sistemi di trasporto sostenibili e l'IBM ha già assegnato \$ 50 milioni  di tecnologie e servizi per aiutare oltre 100 comuni in tutto il mondo.\cite{What_make_city_smart}
\todo{gdm: sono indeciso se parlarne nell'introduzione o in questo chapter}

\section{Fattori e indicatori di una Smart City}
Il Centro per la globalizzazione e la strategia della IESE Business School dell'università di Navarra, in particolare IESE Cities in Motion\footnote{Cities in Motion è una piattaforma di ricerca lanciata congiuntamente dal CGS Center for Globalizaton and Strategy e da IESE Business School entrambi dell'università di Navarra, la cui missione è di creare un approccio innovativo alle città e un nuovo modello urbano per il XXI secolo basato su quattro fattori principali: ecosistema sostenibile, attività creative, uguaglianza tra cittadini e connessione del territorio}, nel 2019 ha stilato una classifica delle dieci migliori città Smart City del mondo basandosi su nove aspetti fondamentali.\cite{iese_cities_2019}
L'obiettivo di questo lavoro è di promuovere i cambiamenti a livello locale e sviluppare idee preziose e strumenti innovativi che porteranno le città ad essere più sostenibili e più intelligenti. 

\todo{gdm: motivare i vari fattoroi e indicatori - perchè questi fanno una sm rispetto ad una city}

\subsection{Economia}
Sono incluse tutte le politiche economiche di una amministrazione pubblica. Tenendo conto del grado di diffusione delle nuovitecnologie e servizi che ne emergono, vengono prese in considerazioni anche i servizi digitali non promossi dalle amministrazioni locali quali Glovo, Uber o MyTaxi.
In figura 1.3 sono rappresentati gli indicatori economici.
\begin{figure}[h!]
	\begin{center}
		\includegraphics[width=320bp]{img/indicatori_economici.png}
		\caption{Indicatori economici}
	\end{center}
\end{figure}

\subsection{Capitale umano}
Le Cità che intendono avviarsi verso una conversione a Smart City devono essere in grado di attrarre e trattenere talenti, creare piani per migliorare l'istruzione e promuovere sia la creatività che la ricerca e migliora il livello di istruzione dei propri cittadini e l'accesso alla cultura (tenendo conto del numero di musei aperti, gallerie d'arte, librerie e spazi dedicato al tempo libero).
In figura 1.1 sono rappresentati gli indicatori relativi al capitale umano.
\begin{figure}[ht]
	\begin{center}
		\includegraphics[width=320bp]{img/indicatori_capitale_umano.png}
		\caption{Indicatori capitale umano}
	\end{center}
\end{figure}

\subsection{Politiche sociali}
Per politiche sociali si intende la capacità di coesistenza tra gruppi di persone con redditi, culture, età e professioni diverse che vivono in una città. Inteso anche come il grado di consenso tra i membri di un gruppo sociale o la percezione dell'appartenenza ad una situazione o progetto comune. Viene considerata come un indice per misurare l'interazione sociale all'interno di un gruppo. Inoltre sono visti come segno positivo il numero di centri sanitari pubblici e privati, intesi come servizi sociali di buon aiuto. Come eventi sfavorevoli giudicati come segno negativo sono inclusi gli ati terroristici subiti negli ultimi tre anni.
In figura 1.2 sono rappresentati gli indicatori relativi alle politiche sociali.
\begin{figure}[ht]
	\begin{center}
		\includegraphics[width=320bp]{img/indicatori_coesione_sociale.png}
		\caption{Indicatori politiche sociali}
	\end{center}
\end{figure}

\subsection{Governance}
Nel lavoro svolto dalla EISE Business School, la governance è considerata come la capacità da parte delle amministrazioni pubbliche locali di amministrare  e di investire i soldi dei contribuenti in politiche volte al miglioramento della qualità della vita del cittadino.
In figura 1.5 vi è la lista degli indicatori per misurare la governance di una Smart City, uni di questi indicatori è la certificazione ISO37120\footnote{La certificazione ISO37120 sancisce un insieme di regole per guidare e misurare le prestazioni dei servizi cittadini e la qualità della vita, questa certificazione può essere applicata a qualsiasi città, comune o governo locale che si impegna a misurare le proprie prestazioni in modo comparabile e verificabile, indipendentemente dalle dimensioni e dalla posizione.} è un forte indicatore in materia di governance per le Smart City, unb altro indicatore è dato dalla percentuale si cittadini occupata nella pubblica amministrazione e nella difesa; formazione scolastica; salute; attività di comunità, servizi sociali e personali; e altre attività.
\begin{figure}[ht]
	\begin{center}
		\includegraphics[width=320bp]{img/indicatori_governance.png}
		\caption{Indicatori di governance}
	\end{center}
\end{figure}

\subsection{Ambiebte}
Sono inclusi tutti quei fattori che migliorano la sostenibilità della città come la creazione di edifici ecologici o di classe energetica A superiore, gestione efficiente dell'acqua e dei rifiuti. Tra gli indicatori sono inclusi la qualità dell'aria e dell'acqua che sono anche indicatori della qualità della vita del cittadino. Ridurre i valori di questi indicatori è sopratutto uno degli obiettivi del protocollo di Kyoto.\footnote{Il protocollo di Kyoto è un trattato internazionale in materia ambientale riguardante il surriscaldamento globale, redatto l'11 dicembre 1997 nella città giapponese di Kyoto da più di 180 Paesi}
In figura 1.4 sono rappresentati gli indicatori ambientali.
\begin{figure}[ht]
	\begin{center}
		\includegraphics[width=320bp]{img/indicatori_ambientale.png}
		\caption{Indicatori ambientali}
	\end{center}
\end{figure}


\subsection{Pianificazione urbana}
Per pianificazione urbana si intende l'insieme dei servizi pubblici volti a migliorare l'abitabilità della città e quindi ad aumentare la qualità della vita del cittadino. Tra gli indicatori, per una buona pianificazione di abitabilità, vi è la presenza di bike sharing, ovvero il numero di punti di stazione di bikesharing presenti nel territorio, altri indicatori sono rappresentati in figura 1.6.
\begin{figure}[ht]
	\begin{center}
		\includegraphics[width=320bp]{img/indicatori_pianificazione_urbana.png}
		\caption{Indicatori di pianificazione urbana}
	\end{center}
\end{figure}

\subsection{Sensibilizzazione internazionale}
Per sensibilizzazione internazionale si intende la capacità di una città di offrire tutti quei servizi strategicy volti a far transitare o a permanere il turista nel territorio. Tra gli indicatori vi è il numero di aereoporti e numero di passeggeri per aereoporto, il numero di congressi che si tengono in città e il . numero di alberghi. Il IESE Cities Motion utilizza i dati di \textit{Sightsmap.com}\footnote{Sightsmap.com è un servizio web che mostra la popolarità di un luogo basandosi sul numero di foto panoramiche scattate.} come indicatore per ottenere il grado di popolarità di un luogo. In figura 1.7 la lista degli altri indicatori.
\begin{figure}[ht]
	\begin{center}
		\includegraphics[width=320bp]{img/indicatori_sensibilizzaqzione_internazinale.png}
		\caption{Indicatori di sensibilizzazione internazionale}
	\end{center}
\end{figure}


\subsection{Tecnologia}
L'aspetto tecnologico, per una città che intende evolversi in Smart City, è il fattore fondamentale da tenere in considerazione. Tra gli indicatori tecnologici che sono visti positivamente in una Smart City sono il numero di Hotspot WiFi\footnote{Un Hotspot è un luogo coperto da una connessione pubblica aperta a tutti.} sparsi per la città, la percentuale di famiglie connesse a internet e la percentuale di famiglie che possiede un pc. L'introduzine nella società di queste tecnologie può causare uno sgradevole fenomeno all'inerno di una Città ovvero può  creare il cosidetto \textit{digital divide}\footnote{Dall'enciclopedia Treccani: il digital devide è un espressione nata in seno all’amministrazione statunitense della presidenza Clinton (1993-2001) per indicare la disparità nelle possibilità di accesso ai servizi telematici tra la popolazione americana e sta ad indicare la consapevolezza globale di una problematica di accesso ai mezzi di informazione e comunicazione da parte di determinate aree geografiche o fasce di popolazion},\cite{problem_smar_city_digitaldivide} è compito delle politiche sociali di una amministrazione pubblica fare in modo che questo fenomeno non si evolvi. \cite{iese_cities_2019}
l'IESE Citie Motion, tra gli indicatori, tiene conto anche dell'attività dei cittadini sui social in particolare Twetter\footnote{Twitter è un servizio di notizie e microblogging fornito dalla società Twitter, Inc. su cui gli utenti postano[2] e interagiscono con messaggi chiamati tweet. La Twitter, Inc. ha sede in San Francisco (Stati Uniti)} e Linkedin\footnote{LinkedIn è un servizio web di rete sociale, gratuito, rivolto principalmente nello sviluppo di contatti professionali e nella diffusione di contenuti specifici relativi al mercato del lavoro (es. motore di ricerca del lavoro, pubblicità aziende, ecc.)}.
In figura 1.8 è rapresentata la tabella completa di tutti gli altri indicatori tecnologicy.
\begin{figure}[ht]
	\begin{center}
		\includegraphics[width=320bp]{img/indicatori_tecnologici.png}
		\caption{Indicatori tecnologici}
	\end{center}
\end{figure}

\subsection{Mobilità e trasporto urbano}
Agevolare il movimento all'interno della città, il numero di mezzi pubblici e il loro accesso è un altro fattore importante che non può mancare in una Smart City. Gli indicatori di questa categoria sono il numero stazioni della metropolitana, l'indice del traffico, qui ritroviamo dinuovo il numero di Bikesharing ecc.
In figura 1.9 tutti gli altri indicatori.
\begin{figure}[ht]
	\begin{center}
		\includegraphics[width=320bp]{img/indicatori_trasporti_urbani.png}
		\caption{Indicatori di mobilità e trasporto urbano}
	\end{center}
\end{figure}

\section{Ranking delle Smart City}
Nel Ranking stilato dall'IESE Business School - IESE Cities in Motion, su un campione di 174 città e tenendo conto di tutti i fattori e gli indicatori sopra descritti vi è al primo Posto Londra con un valore CIMI\footnote{Valore CIMI è il punteggio assegnato tenendo conto di tutti i fattori} di 100 e una performance\footnote{Il valore CIMI viene ragruppato in cinque categorie: Hight con un cimi superiore a 90; Relatively Hight tra 60 e 90; Medium tra 45 e 60; Low 15 e 45 e Very Low inferiore a 15 } Hight a seguire New York city Amsterdam, tra le città italiane troviamo Milano al quarantunesima posizione con un CIMI di 65.94, l'unica con una performance Relatively Hight a seguire con una performance Medium vi è al settantacinquesimo posto vi è Roma con CIMI di 59.09, Firenze con 49.54, Torino con 49.51 e Napoli con 46.62.


\section{Oggetti in gioco}
In questa sezione vengono descritti i diversi aspetti tecnologici, che vanno dalla raccolta delle informazioni fino al aspetto tecnologico con cui viene trattato il dato.

\subsection{Architettura delle Smart City}
Una Smart City è costituita dai seguenti livelli gerarchici:\cite{smart:city_cybersecurity_privacy_cap4}
\begin{itemize}
  \item \textbf{Livello 1 - Ambiente} include tutte le caratterisciche ambientali della cittò;
  \item \textbf{Livello 2 - Infrastruttura Hardware (no ICT based)} è costituita da tutte le infrastrutture urbane quali edifici, ponti, strade, viadotti, acqueddotti ecc.;
  \item \textbf{Livello 3 - Infrastruttura Hardware (ICT based)} riguarda tutto l'hardware con cui vengono raccolte e prodotte tutte le informazioni come l'infrastruttura IoT, sensori, attuatori, reti di comunicazione, ecc.;
  \item \textbf{Livello 4 - Smart Service} sono tutti quei servizi direttamente prodotti dall'infrastruttura hardware e software come smart safety, intelligent transportation, smart government, smart water, ecc.;
  \item \textbf{Livello 5 - infrastruttura Software} costituisce l'insieme di persone fisiche o utenti finale della città che usufruiscono dei servizi smart.
  
\end{itemize}


\subsection{Internet of Things}
La raccolta di informazioni all'interno di una Smart City avviene attraverso una serie di oggetti chiamati sensori\footnote{I sensori sono dispositivo meccanico, elettronico o chimico, che in apparecchiature o meccanismi rileva i valori di una grandezza fisica e ne trasmette le variazioni a un sistema di misurazione o di controllo.}, sparsi in giro per la città. Questi sonseori sono connessi a internet attraverso specifici protocolli di comunicazione. Questo approccio alla raccolta di dati viene chiamato IoT acronimo di Internet of Things (internet delle cose).
All'interno di una Smart City si possono trovare sensori per il rilevamento della qualità dell'aria come il pm10 e pm2.5\footnote{PM sta a indicare il particulate matter e sta ad indicare l'insieme delle sostanze sospese in aria, precisamente il particolato è l'inquinante oggi più frequente nelle aree urbane, ed è composto di particelle solide o liquide disperse nell'atmosfera.}, il gps\footnote{Il gps, acronimo di global positioning system, sistema di posizionamento globale, è un sistema di posizionamento e navigazione satellitare militare statunitense che una rete dedicata di satelliti artificiali in orbita attorno alla terra.}, questo sensore vienen sopratutto utilizzato per monitorare gli spostamenti dei vai mezzi pubblici oppure bike e car sharing.

\subsection{Criteri architetturali di una smart city}
riferimento\cite{smart_city_architetture}
FIWARE come framework aperto 

\subsection{Partecipazione e collaborazione}
Per partecipazione e collaborazione si intende l’iniziativa spontanea del cittadino nel mantenimento dello stato ottimale dei servizi o del arredo urbano attraverso appositi strumenti forniti dall'amministrazione locale. Questo può avvenire fornendo al cittadino strumenti di ticketing per segnalare eventuali guasti o danni ad impianti pubblici come installazione di pubblica illuminazione o strade rotte. In questo modo il comune o chi di dovere abbia conoscenza della manutenzione richiesat all'interno della città.

In italia, la società Citelum SA, società francese del gruppo EDF che si occupa di efficentamento energetico e pubblica illuminazione, ha distribuito ad oltre duecento comune del centro nord Itala, un sistema di ticketing, rivolto ai cittadini, per la segnalazione di guasti o anomalie agli impianti di pubblica illuminazione e alla segnaletica semaforica, tramite l'utilizzo del proprio smartphone. Ogni istallazione luminosa o punto luce (palo della luce o semaforo), è dotata di una etichetta identificativa con relativo qr-code\footnote{Il qr-code è un codice a barre bidimensionale (o codice 2D), composto da moduli neri disposti all'interno di uno schema bianco di forma quadrata, impiegato tipicamente per memorizzare informazioni generalmente destinate a essere lette tramite uno smartphone.}, l'utente, dopo aver scansionato il qrcode, viene reindirizzato alla pagina della segnalazione, in figura 1.10 un esempio di segnalazione preso da un fermo immagine del video di presentazione della Smart City di Lonato promossa sempre da Citelum SA\cite{citelum_smart_city_lonato}.
\begin{figure}[h!]
    \centering
        \includegraphics[width=150bp]{img/faro/faro_qrcode.png}
            \quad
        \includegraphics[width=150bp]{img/faro/faro_openpage.png}
            \quad
        \includegraphics[width=150bp]{img/faro/faro_segnalazione_1.png}
            \quad
        \includegraphics[width=150bp]{img/faro/faro_segnalazione_2.png}
    \caption{Esempio di apertura di un ticketing}
\end{figure}
Una volta scansionato il qr-code, con un apposita applicazione per la lettura dei qr-code, il sistema chiederà all'utente di essere reindirizzato ad una pagina web, tale pagina è composta da una mappa georeferenziata con il punto luce, i dati relativi alla composizione dell'impianto semaforico e un form di registrazione dove vene richiesto di inserire:
\begin{wrapfigure}{r}{0.3\textwidth}
    \includegraphics[width=0.40\textwidth]{img/faro/qrlonato2.jpg}
\end{wrapfigure}
\begin{itemize}
  \item \textbf{Nome del segnalante}
  \item \textbf{Email del segnalante}
  \item \textbf{Tipo guasto scelto da una lista}
  \item \textbf{Allegare una foto dell'impianto} (Opzionale)
  \item \textbf{Spunta sull'informativa UE2016/679 e D.lgs 196/2003}
\end{itemize}

L'esempio di for viene mostrato nell'immagine affianco. Dopo aver compilato il form, al click su "Invia", l'utente invierà una segnalazione alla società incaricata alla manutenzione della pubblica illuminazione relativo a quel comune.


