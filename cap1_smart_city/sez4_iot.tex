\section{Internet of Things}
La raccolta di informazioni all'interno di una Smart City avviene attraverso una serie di oggetti chiamati sensori\footnote{I sensori sono dispositivo meccanico, elettronico o chimico, che in apparecchiature o meccanismi rileva i valori di una grandezza fisica e ne trasmette le variazioni a un sistema di misurazione o di controllo.}, sparsi in giro per la città. Questi sonseori sono connessi a internet attraverso specifici protocolli di comunicazione. Questo approccio alla raccolta di dati viene chiamato IoT acronimo di Internet of Things (internet delle cose).
All'interno di una Smart City si possono trovare sensori per il rilevamento della qualità dell'aria come il pm10 e pm2.5\footnote{PM sta a indicare il particulate matter e sta ad indicare l'insieme delle sostanze sospese in aria, precisamente il particolato è l'inquinante oggi più frequente nelle aree urbane, ed è composto di particelle solide o liquide disperse nell'atmosfera.}, il gps\footnote{Il gps, acronimo di global positioning system, sistema di posizionamento globale, è un sistema di posizionamento e navigazione satellitare militare statunitense che una rete dedicata di satelliti artificiali in orbita attorno alla terra.}, questo sensore vienen sopratutto utilizzato per monitorare gli spostamenti dei vai mezzi pubblici oppure bike e car sharing.
